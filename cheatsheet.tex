\documentclass[12pt,english,oneside]{scrbook}
\usepackage{ae,aecompl}
\usepackage[T1]{fontenc}
\usepackage[latin9]{inputenc}
\usepackage{geometry}
%\geometry{verbose,tmargin=1in,bmargin=1in,lmargin=1in,rmargin=1in}
\setlength{\parskip}{\smallskipamount}
\setlength{\parindent}{0pt}
\usepackage{amsmath}
\usepackage{amsthm}
\usepackage{amssymb}
\usepackage{setspace}
\setstretch{1.1}

\makeatletter
\numberwithin{equation}{chapter}
\theoremstyle{definition}
\newtheorem*{example*}{\protect\examplename}
\theoremstyle{plain}
\newtheorem{thm}{\protect\theoremname}[section]

\@ifundefined{date}{}{\date{}}
\AtBeginDocument{
\addtolength{\abovedisplayskip}{0ex}
\addtolength{\abovedisplayshortskip}{0ex}
\addtolength{\belowdisplayskip}{0ex}
\addtolength{\belowdisplayshortskip}{0ex}
}

\newtheorem{theorem}{Theorem}[section]
\newtheorem{lemma}{Lemma}[section]
\newtheorem*{corollary*}{Corollary}

\makeatother

\usepackage{babel}
\providecommand{\examplename}{Example}
\providecommand{\theoremname}{Theorem}

\begin{document}


\title{Combinatorics: Cheatsheet}
\author{Ronald Mangang}

\maketitle
\tableofcontents
\vfill
These notes closely follow \textit{Principles and Techniques of Combinatorics}.
\thispagestyle{empty}
\newpage
\pagenumbering{arabic}


\chapter{Inclusion-Exclusion}

\paragraph{Inclusion-exclusion principle.}




Suppose $A_1, A_2, \ldots, A_n$ are finite sets then
\begin{align*}
  |\bigcup_{i=1}^n A_i | &= \sum_{i=1}^n|A_i|-\sum_{i<j}^n |A_i \cap A_j | +
  \cdots + \\
  &(-1)^{n+1} |A_1 \cap A_2 \cap \ldots \cap A_n |.
\end{align*}

\paragraph*{Generalized inclusion-exclusion principle.}
Let $S$ is an $n$-element set and $\{P_1,P_2, \ldots,P_q\}$ be a set of $q$ properties which may be satisfied by elements of $S$. We define
\[
  \omega(P_{i1}P_{i2}\cdots P_{im}) = |\cap_{j=1}^m A_{ij} |
\]
where $A_{ij} \in S$ is the set of elements having property $P_{ij}$. We define
\[
  \omega(m) = \sum(\omega(P_{i1}P_{i2}\cdots P_{im}).
\]
That is, $\omega(m)$ is the number of elements of $S$ having at least $m$ properties.\\

We define
\[
  E(m) = \text{number of elements of $S$ having exactly $m$ properties.}
\]
Then
\begin{align*}
  E(m) =& \omega(m) - \binom{m+1}{m}\omega(m+1)+\binom{m+2}{m}\omega(m+2) \\
        & \cdots + (-1)^{q-m}\binom{q}{m}\omega(q) \\
  =& \sum_{k=m}^q (-1)^{k-m}\binom{k}{m}\omega (k).
\end{align*}

\begin{proof}
  Will be updated.
\end{proof}

We define $\omega(0) = |S|$. \\

As a special case, we have
\begin{align*}
  E(0) &= \omega(0) - \omega(1) + \cdots + (-1)^q \omega(q) \\
       &= \sum_{k=0}^q (-1)^k \omega(k).
\end{align*}

Let $A_1, A_2, \ldots, A_q$ be any $q$ subsets of a finite set $S$. Then
\begin{align*}
  E(0)=&|\bar A_1 \cap \bar A_2 \cap \cdots \cap \bar A_q | \\
  =&|S| - \sum_{i=1}^q |A_i| + \sum_{i<j}^q |A_i \cap A_j | - \sum_{i<j<k}^q |A_i \cap A_j \cap A_k | + \cdots \\
  &(-1)^q |A_1 \cap A_2 \cap \cdots \cap A_q |
\end{align*}
which leads to the familiar exclusion-inclusion principle.

\paragraph{Stirling numbers of the second kind.}

The number of onto functions from $A$ to $B$ where $|A|=n$ and $|B|=m$ is
\begin{equation}
  F(n,m) = \sum_{k=0}^m (-1)^k \binom{m}{k} (m-k)^n.
\end{equation}

\begin{proof}
  Will be updated.
\end{proof}

$F(n,m)$ is the number of ways of distributing $n$ distinct objects into $m$ distinct boxes so that no box is empty. Then it follows that
\begin{equation}
  F(n,m) = m! S(n,m)
\end{equation}
which gives a formula to compute $S(n,m)$.

\paragraph{Permutations with fixed positions.}

Let $0\leq k \leq r \leq n$. We define $D(n,r,k)$ to be number of $r$-permutations of $\{1,2,\ldots,n\}$ that have exactly $k$ fixed positions. From the generalized inclusion-exclusion principle, we can obtain
\begin{equation}
  D(n,r,k) = \frac{\binom{r}{k}}{(n-r)!} \sum_{i=0}^{r-k} (-1)^i \binom{r-k}{i}(n-k-i)!.
\end{equation}

\begin{proof}
  Will be updated.
\end{proof}

\paragraph{Euler $\varphi$-function.}

We define $\varphi (n)$ to be the number of integers between $1$ and $n$ which are co-prime to $n$. Let $n\in \mathbb{N}$ and let
\[
  n = p_1^{m_1}\cdot p_2^{m_2}\cdots p_k^{m_k}
\]
be its prime factorization. Using the generalized inclusion-exclusion principle, we can obtain
\begin{equation}
  \varphi(n) = n\left(1-\frac{1}{p_1}\right)\left(1-\frac{1}{p_2}\right)\cdots \left(1-\frac{1}{p_k}\right) = n\prod _{i=1}^k \left(1-\frac{1}{p_i}\right).
\end{equation}

\begin{proof}
  Will be updated.
\end{proof}




\end{document}


